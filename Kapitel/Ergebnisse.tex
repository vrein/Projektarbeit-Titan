\chapter {Ergebnisse}

\section{$\alpha_p$-Studie (PH)}

Im Rahmen der Alpha-P Studie wurden zunächst 3 Proben bei verschiedenen Temperaturen unterhalb der Beta-Transus Temperatur (995$^\circ$C für Ti-6242) wärmebehandelt. Ziel war die Einstellung einer bimodalen Mikrostruktur, sowie die Bestimmung des Alpha-Primär-Volumenanteils. Es wurden 3 Temperaturen (990$^\circ$C, 975$^\circ$C und 960$^\circ$C) ausgewählt, bei denen die Proben für eine Stunde im Ofen Wärmebehandelt und anschließend luftabgekühlt wurden. Zusätzlich wurde eine Probe überhalb der Beta-Transus-Temperatur bei 1015$^\circ$C für 30 Minuten geglüht und anschließend in Wasser abgeschreckt, um zum Vergleich der verschiedenen Mikrostrukturen, ein vollmartensitisches Gefüge einzustellen. Die Auswertung dieser Proben unter dem Lichtmikroskop sind in Abbildung 8 aufgeführt. 

\begin{figure}[h]
	\centering
	\includegraphics[width=0.9\linewidth]{"Bilder/Abbildung 8"}
	\caption[Abbildung 8]{Mikrostrukturen der verwendeten Ti-6242 Legierung vor und nach der ersten Wärmebehandlung bei verschiedenen Temperaturen, oben links: Mikrostruktur vor Wärmebehandlung, oben mitte: 960$^\circ$C/1h/AC, oben rechts: 975$^\circ$C/1h/AC, unten links: 983$^\circ$C/1h/AC, unten mitte: 990$^\circ$C/1h/AC, unten rechts: 1015$^\circ$C/30min/WQ vollmartensitisches Gefüge}
	\label{fig:abbildung-8}
\end{figure}

Die Ergebnisse der Bestimmung des Alpha-P Volumenanteils mittels Bildbearbeitungsprogramm sind in Tabelle 4 aufgeführt. Laut Lütjering und Williams liegt der optimale Primär Alpha Volumenanteil zur Steigerung der Zugfestigkeitswerte zwischen 10 und 20 \% [2]. Da die bis dahin erstellten Proben mit ihren Primär Alpha Volumenanteilen außerhalb dieses Bereiches lagen, wurde eine weitere Probe bei 983$^\circ$C für eine Stunde geglüht und anschließend luftgekühlt. Die resultierende Mikrostruktur ist ebenfalls in Abbildung 8 aufgeführt.

\begin{table}[h]
	\centering
	\begin{tabular}{|c|c|}
		\hline 
		& Primär-$\alpha$ in \% \\ 
		\hline 
		AR & 62 \\ 
		\hline 
		960$^\circ$C/1h/AC & 37 \\ 
		\hline 
		975$^\circ$C/1h/AC & 26 \\ 
		\hline 
		983$^\circ$C/1h/AC & 16 \\ 
		\hline 
		990$^\circ$C/1h/AC & 9 \\ 
		\hline 
		1015$^\circ$C/30min/WQ & 0 \\ 
		\hline 
	\end{tabular} 
	\caption{Primär-$\alpha$ Volumenanteile der ersten Wärmebehandlungen mit einer durchnittlichen Abweichung von 3\%}
	\label{Tabelle 4}
\end{table}

Die Auswertung hat ergeben, dass der angestrebte Primär-$\alpha$ Volumenanteil mit der Wärmebehandlung bei 983$^\circ$C für 1 Stunde mit anschließender Luftkühlung erreicht wurde. Die vollmartensitische Probe hat wie erwartet keinen sichtbaren Primär-$\alpha$ Anteil aufgewiesen.


Desweiteren wurde an der ersten Probenreihe eine Härteprüfung durchgeführt. Die Ergebnisse sind zusammen mit der Standardabweichung in Tabelle 5 aufgeführt. 


\begin{table}[h]
	\centering
	\begin{tabular}{|c|c|c|}
		\hline 
		& Härte in HV &  Std.-abw. \\ 
		\hline 
		AR & 331 & 2.45 \\ 
		\hline 
		960$^\circ$C/1h/AC & 345 & 2.83 \\ 
		\hline 
		975$^\circ$C/1h/AC & 344 & 2.80 \\ 
		\hline 
		983$^\circ$C/1h/AC & 344 & 1.84 \\ 
		\hline 
		990$^\circ$C/1h/AC & 350 & 4,74 \\ 
		\hline 
		1015$^\circ$C/30min/WQ & 403 & 3.94 \\ 
		\hline 
	\end{tabular} 
    \caption{Härtewerte der ersten Probenreihe in HV und ihre Standardabweichung}
    \label{Tabelle 5}
\end{table}

Nach der ersten Wärmebehandlung war bei den bimodalen Mikrostrukturen keine wesentliche Härtesteigerung messbar.

\pagebreak

\section{Short Time Duplex Heat Treatment (STDA - short Time Duplex Anneal) (PH)}

Im nächsten Schritt wurde versucht, die STDA Wärmebehandlung von der $\alpha$+$\beta$ Legierung Ti-6Al-4V auf die Near-$\alpha$ Legierung Ti-6AL-2Sn-4Zr-2Mo zu übertragen. Ziel war es zunächst in einem zweiten Prozessschritt Martensit im transformierten Beta zu erzeugen. Dafür wurden die Proben mit bimodalen Mikrostrukturen aus der ersten Wärmebehandlung erneut bei 930$^\circ$C im Ofen für 8 Minuten geglüht und anschließend wassergekühlt. Die Auswertung unter dem Lichtmikroskop ist in Abbildung 9 zusammengefasst.

\begin{figure}[h]
	\centering
	\includegraphics[width=0.9\linewidth]{"Bilder/Abbildung 9"}
	\caption[Abbildung 9]{Mikrostrukturen von Ti-6242 nach dem zweiten Prozessschritt}
	\label{fig:abbildung-9}
\end{figure}

Nach dem zweiten Prozessschritt konnte keine Veränderung der Mikrostrukturen unter dem Lichtmikroskop festgestellt werden. Daher wurden die Proben unter dem Rasterelektronenmikroskop (REM) näher untersucht, um festzustellen, ob sich Martensit im transformierten Beta geformt hat. Die Ergebnisse sind in Abbildungen 10-13 aufgeführt.

\pagebreak

\begin{figure}[!]
	\centering
	\includegraphics[width=0.9\linewidth]{"Bilder/Abbildung 10"}
	\caption[Abbildung 10]{960$^\circ$C/1h/AC + 930$^\circ$C/8min/WQ, REM unter verschiedenen Auflösungen, Randbereich}
	\label{fig:abbildung-10}
\end{figure}

\begin{figure}[!]
	\centering
	\includegraphics[width=0.9\linewidth]{"Bilder/Abbildung 11"}
	\caption[Abbildung 11]{975$^\circ$C/1h/AC + 930$^\circ$C/8min/WQ, REM unter verschiedenen Auflösungen, Randbereich}
	\label{fig:abbildung-11}
\end{figure}

\begin{figure}[!]
	\centering
	\includegraphics[width=0.9\linewidth]{"Bilder/Abbildung 12"}
	\caption[Abbildung 12]{983$^\circ$C/1h/AC + 930$^\circ$C/8min/WQ, REM unter verschiedenen Auflösungen, Randbereich}
	\label{fig:abbildung-12}
\end{figure}

\begin{figure}[!]
	\centering
	\includegraphics[width=0.9\linewidth]{"Bilder/Abbildung 13"}
	\caption[Abbildung 13]{990$^\circ$C/1h/AC + 930$^\circ$C/8min/WQ, REM unter verschiedenen Auflösungen, Randbereich}
	\label{fig:abbildung-13}
\end{figure}

In den Abbildungen 10-13 ist zu erkennen, dass lediglich die Proben der Temperaturenreihe mit 960$^\circ$C und 990$^\circ$C ansatzweise Martensit im Randbereich aufwiesen. Die Proben der Temperaturen mit 975$^\circ$C und 983$^\circ$C zeigten keine Anzeichen von Martensitbildung.

Die Härteprüfung dieser Probenreihe ist in Tabelle 6 zusammengefasst, zeigt jedoch bei keiner Probe eine sichtbare Härtesteigerung.

\begin{table}[h]
	\centering
	\begin{tabular}{|c|c|c|}
		\hline 
		& Härte in HV &  Std.-abw. \\ 
		\hline 
		960$^\circ$C/1h/AC + 930$^\circ$C/8min/WQ & 350 & 2.99 \\ 
		\hline 
		975$^\circ$C/1h/AC + 930$^\circ$C/8min/WQ & 345 & 3.94 \\ 
		\hline 
		983$^\circ$C/1h/AC + 930$^\circ$C/8min/WQ & 349 & 3.19 \\ 
		\hline 
		990$^\circ$C/1h/AC + 930$^\circ$C/8min/WQ & 352 & 4.51 \\ 
		\hline 
    \end{tabular} 
	\caption{Ergebnisse der Härteprüfung der zweiten Probenreihe}
	\label{Tabelle 6}
\end{table}





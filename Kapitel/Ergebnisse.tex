\chapter {Ergebnisse}

\section{$\alpha_p$-Studie}

\section{Martensit-Bildung}

\section{Martensit-Zerfall}



Für den Martensit Zerfall sind folgenden Werte in HV gemessen worden}
	Inhalt...
\end{rden}
580° C / 8 min / WQ	393 HV		Std. Abweichung von 	2.02
580° C / 16 min / WQ	392 HV		Std. Abweichung von 	4.15
610° C / 8 min / WQ	399 HV		Std. Abweichung von 	2.32
610° C / 16 min / WQ	392 HV		Std. Abweichung von 	2.57
983°C/1h/WQ+610°C/16min/AC	405 HV		Std. A. 	6.22
983°C/1h/WQ+610°C/30min/AC	400 HV		Std. A. 	2.81

Man sieht eine offensichtliche Erhöhung der Festigkeit. Das beste Versuchsprobe aus der 2. Wärmebehandlung ist bei 950° C / 16 min / WQ mit 376 HV gemessen worden. Man beobachtet eine Erhöhung von 17 bis 23 HV vom zweiten auf dritten Schritt. Damit kann man zufriedenweise sagen es eine gute Erhöhung der Festigkeit bei Legierungsmaximierung findet. Nur daraus kann man mit große Wahrscheinlichkeit feststellen ((behaupten)) dass eine Martensitische Zerfall passiert ist. 
Für den Parallelversuch ist keine Erhöhung zu sehen. Beim Glühen der Probe bei 983° C / 1h / Wasser gekühlt ist eine Härte von 405 HV. Leider ist bei den diesen Schritt die Festigkeit nicht gestiegen. Grund dafür könnte ja sein, dass es nicht genug Zeit gelassen worden ist. Es handelt sich um eine globale Struktur, also es findet sich viel mehr Martensit als im Duplex Legierung. 

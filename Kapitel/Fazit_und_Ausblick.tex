\chapter{Fazit und Ausblick}

(ZB,TJ,PH,VR)

Im Rahmen dieser Projektarbeit wurden 2 Wärmebehandlungsstrategien zur Optimierung der Zugfestigkeit der Legierung Ti-6242 betrachtet. Dabei sollte eine Bruchdehnung von mindestens 10\% beibehalten werden. 
Sowohl bei der TS-STDA als auch der STQA kam es zu einer Festigkeitssteigerung gegenüber den Ausgangsproben.
Nach beiden Strategien fiel die Bruchdehnung der Proben unter den geforderten Wert von 10\%. Es konnte gezeigt werden, dass die für Ti-64 entwickelte Wärmebehandlung auf Ti-6242 übertragen werden kann. Nach Anpassung der Anlasstemperatur und Haltezeit konnte auch bei Ti-6242 lokal Martensit gebildet werden, der anschließend zum Zerfall gebracht wurde. Die stark abweichende Gefügestruktur der zweiten Strategie macht einen Vergleich mit denselben Parametern aus der ersten Strategie nicht aussagekräftig.
Um die Duktilität und dadurch die Bruchdehnung zu erhöhen, können mehrere Parameter betrachtet werden. Um die  Duktilitätszunahme durch weiteren Martensitzerfall zu untersuchen, kann die Glühtemperatur und Haltezeit im letzten Wärmebehandlungsschritt angepasst und optimiert werden. Zusätzlich wäre eine weitere Untersuchung mit unterschiedlichen bi-modalen Ausgangsgefügen sinnvoll.

Martensitische Gefüge sind in Hochtemperaturgebieten nur begrenzt einsetzbar, da ihre Dekomposition zu einer signifikanten Änderung der mechanischen Eigenschaften führt.

Für den Einsatz in der Luftfahrt müssen weiterführende Untersuchungen durchgeführt werden. Dazu könnten weitere wichtige mechanische Kennwerte durch Zugversuche bei hohen Temperaturen sowie Dauerfestigkeitsversuche ermittelt werden.
Des Weiteren wären Untersuchungen zum Kriech- und Ermüdungsverhalten sinnvoll.
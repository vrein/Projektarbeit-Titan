\chapter{Einleitung (VR)} 

Titan nimmt durch seine herausragenden Eigenschaften eine immer stärkere Rolle im Bereich der Luft- und Raumfahrttechnik ein. Dazu zählen die hohe spezifische Festigkeit, Korrosions"~, Kriech- und Temperaturbeständigkeit sowie die Dauerfestigkeit \cite{Lutjering.2007}. Diese sind abhängig von der Mikrostruktur des Werkstoffs, die maßgeblich durch die Legierungszusammensetzung und thermomechanische Behandlung bestimmt wird. Dadurch ist eine Vielzahl an Anwendungen in verschiedenen Bereichen des Flugzeugbaus möglich \cite{M.J.Donachie.2010}.

Die bekannteste und am besten erforschte Legierung ist dabei Ti-6Al-4V (Ti-64), die bereits seit den 1950er Jahren erforscht wird und die seitdem eine dominierende Rolle auf dem Markt eingenommen hat. Ti-64 wird insbesondere in dynamisch belasteten Bauteilen wie Fanschaufeln in Flugtriebwerken und Verbindungselementen wie Nieten und Bolzen verwendet. 
Für eine verbesserte Wärmebeständigkeit bei ähnlichen Schmiede- und Verformungseigenschaften wurde Ti-6Al-2Sn-4Zr-2Mo (Ti-6242) entwickelt. Die mögliche Einsatztemperatur dieses Werkstoffs liegt bei bis zu $550^\circ $C im Vergleich zu $350^\circ$C für Ti-64 und erlaubt somit den Einsatz in temperaturempfindlichen Bereichen \cite{Lutjering.2007}.
 
Ziel dieser Projektarbeit ist die Zugfestigkeit der Legierung Ti-6242 durch eine Wärmebehandlung zu maximieren, während eine Bruchdehnung von mindestens 10\% beibehalten wird. Um diese Zielvorgaben zu erreichen, wird eine Wärmebehandlung für Ti-64, die eine Steigerung der Zugfestigkeit von bis zu 25\% verspricht, betrachtet und für die Legierung Ti-6242 übertragen und angepasst \cite{Morita.2005}. Diese Wärmebehandlung ist unter dem Namen \textit{Three Step Short Time Duplex Anneal} (TS-STDA) bekannt. Dazu wird zuerst eine $\alpha_p$-Studie durchgeführt, in der der Zusammenhang zwischen dem $\alpha_p$-Volumenanteil und der Härte untersucht wird. Anschließend wird die Martensitbildung und der Martensitzerfall betrachtet. Parallel wurde als zweite Strategie die in der Industrie weitverbreitete \textit{Solution treatment and quenching + Aging} (STQA) untersucht. Die Ergebnisse dieser beiden Strategien werden im Rahmen dieser Studie gegenübergestellt und verglichen.

%Eine für Ti-64 bereits erforschte Wärmebehandlung ist die \textit{Three-Step Short-Time-Duplex-Anneal} (TS-STDA), bei der eine Steigerung der Zugefestigkeit von bis zu 25\% möglich ist. Vorteilhaft sind die kürzeren Haltezeiten und die geringe Duktilitätsabnahme verglichen mit vollmartensitischen Gefügen. Für einen erfolgreichen Transfer der Wärmeübertragung müssen Faktoren wie die geringere $\beta$-stabilisierende Wirkung und unterschiedliche Diffusionsrate der Legierungselemente und die niedrigere Martensitstart-Temperatur ($M_s$) des Ti-6242 beachtet werden. 

Die Kennwerte, die für den Vergleich mit vollmartensitischen Gefügen, konventionellen Wärmebehandlungen und zur Kontrolle der Zielparameter nötig sind, werden mithilfe von Licht- und Elektronenmikroskopie, der Vickers-Härteprüfung und Zugversuchen bestimmt.

%Die Möglichkeit durch die Mikrostruktur die Eigenschaften des Titans bedarfsgerecht einzustellen ist der allotropen Phasenumwandlung bei $882\circ C$ von einem kubisch-raumzentrierten (krz) Gitter ($\beta$-Phase) zu einer hexagonal annähernd dichtesten Packung (hdp) ($\alpha$-Phase) bei Raumtemperatur geschuldet. Dabei kann durch Zugabe von Legierungselementen die Übergangstemperatur der allotropen Umwandlung abgesenkt ($\beta$-stabilisierend) oder angehoben werden ($\alpha$-stabilisierend). Die hier betrachtete Legierung Ti 6Al 2Sn 4Zr 2Mo (Ti-6242) hat durch den $\beta$-isomorphen Stabilisator Molybdän eine eine geringere $\beta$-Transus-Temperatur mit steigendem Molybdänanteil, sodass die $\beta$-Phase auch bei Raumtemperatur stabil sein kann. %




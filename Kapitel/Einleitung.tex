\chapter{Einleitung} 
Titan nimmt durch seine herausragenden Eigenschaften eine immer stärkere Rolle im Bereich der Luft- und Raumfahrttechnik ein. Dazu zählen Eigenschaften wie die hohe spezifische Festigkeit, Korrosions- und Temperaturbeständigkeit sowie die Dauerfestigkeit, Kriechbeständigkeit und Rissausbreitung. Diese sind abhängig von der Mikrostruktur des Werkstoffs, die maßgeblich durch die Legierungszusammensetzung und thermomechanische Behandlung bestimmt wird. Dadurch wird eine Vielzahl an Anwendungen in verschiedensten Bereichen des Flugzeugbaus, wie der Flugzeugzelle, dem Fahrwerk und den Triebwerken ermöglicht. Trotz der durch den großen Produktionsaufwand relativ hohen Werkstoffkosten hat sich Titan durch die vorteilhafte Kombination seiner Eigenschaften in der militärischen und zivilen Luftfahrt durchgesetzt. \\ 
%Die Möglichkeit durch die Mikrostruktur die Eigenschaften des Titans bedarfsgerecht einzustellen ist der allotropen Phasenumwandlung bei $882\circ C$ von einem kubisch-raumzentrierten (krz) Gitter ($\beta$-Phase) zu einer hexagonal annähernd dichtesten Packung (hdp) ($\alpha$-Phase) bei Raumtemperatur geschuldet. Dabei kann durch Zugabe von Legierungselementen die Übergangstemperatur der allotropen Umwandlung abgesenkt ($\beta$-stabilisierend) oder angehoben werden ($\alpha$-stabilisierend). Die hier betrachtete Legierung Ti 6Al 2Sn 4Zr 2Mo (Ti6242) hat durch den $\beta$-isomorphen Stabilisator Molybdän eine eine geringere $\beta$-Transus-Temperatur mit steigendem Molybdänanteil, sodass die $\beta$-Phase auch bei Raumtemperatur stabil sein kann. %

Die gängigste Wärmebehandlung für Ti6242 ist ein Glühen im Zweiphasenfeld knapp unter der $\beta$-Transus-Temperatur und anschließendes Abkühlen an der Luft. Dadurch entsteht ein bimodales Gefüge bestehend aus Primär-$\alpha$-Körnern ($\alpha_p$) und lamellaren transformierten $\beta$-Körnern. 

Ein Großteil der in der Luftfahrt eingesetzten Titanlegierungen bestehen aus zwei Phasen, der $\alpha$ und $\beta$-Phase in unterschiedlichen Volumenanteilen und Morphologien. Die bekannteste und am besten erforschte Legierung ist dabei Ti 6Al 4V (Ti64), an der bereits seit den 50er Jahren geforscht wird. Ti64 wird insbesondere in dynamisch belasteten Bauteilen wie Fanschaufeln in Flugtriebwerken und Verbindungselementen wie Nieten und Bolzen. 

Ziel dieser Projektarbeit ist die Zugfestigkeit der Legierung Ti6242 durch eine Wärmebehandlung zu maximieren, während eine Bruchdehnung von mindestens 10\% beibehalten wird.  

\begin{table}[h] 
	\centering 
	\begin{tabular}{lr} 
		
		Name & \hspace{0.5cm} Initialen\\ 
		\hline 
		Ziad Ben Hadj Salem & ZB\\
		Thiago Coelho Jordao & TJ\\
		Patrick Hartmann & PH\\
		Viktor Rein & VR\\
		\hline
		
	\end{tabular} 
	\caption{Initialen der beteiligten Personen} 
	\label{tab:initialien} 
\end{table} 

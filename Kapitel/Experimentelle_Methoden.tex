\chapter{Experimentelle Methoden}

\section {Metallografische Präparation}


\subsection{Ofenbehandlung }
Alle Proben werden unter Normalatmosphäre wärmebehandelt. Da für die mechanischen Eigenschaften 



\subsection{Trennen}
Die wärmebehandelten Proben werden in der Mitte mit einer Siliziumkarbid-Scheibe unter ständigem Kühlmittelfluss getrennt (Trennmaschine Jean Wirtz CUTO 20). Durchgehende Kühlung  verhindert eine zusätzliche, ungewollte Gefügeveränderung an der Schnittfläche während des Trennvorgangs.


\subsection{Einbetten}
Die getrennten Proben werden in Warmeinbettpressen (Buehler Simplimet Mounting Press 1000/4000) für besseres Handling und Stützung der Randzonen eingebettet. Beim Warmeinbetten wird mit Hilfe von Druck und Temperatur die Probe in ein Kunststoffgranulat eingebettet. Vorteile des Warmeinbettens sind die hohe Härte und Spaltfreiheit des Einbettmaterials. Dabei wird Epomet als erste Schicht im Bereich der Probenoberfläche benutzt und für die oberflächenfernen Bereiche Bakelit, da Epomet eine bessere Spaltfüllung hat. Das Warmeinbetten erfolgte bei $180 \deg C$ und $3 bar$. 
Die fertig eingebetteten Proben werden entgratet und auf der Seite der Probe mit einer Fase versehen.  


\subsection{Schleifen/Polieren}

Die Trennfläche der Proben wird in Vorbereitung auf die Ätzung der Oberfläche geschliffen und poliert. Ziel ist eine Oberfläche, die frei von Riefen und Fremdpartikeln ist. Als Schleif-/Poliergerät wurde ein ATM Saphir 550 benutzt.
Im ersten Schritt werden die Proben mit steigender Körnung im Gegenlauf geschliffen und dabei wassergekühlt (\ref{tab:Schleifstufen}). 

\begin{table}[h]
	\centering
	\begin{tabular}{|c|c|c|c|c|c|c|c|c|}
		
		\hline 
		Körnung (FEPA P) & 180 & 240 & 320 & 400 & 600 & 800 & 1200 & 2500 \\ 
		\hline 
		Zeit (min:s) & 0:30 & 1:00 & 1:30 & 2:00 & 2:30 & 3:00 & 3:30 & 4:00 \\ 
		\hline 
		Anpressdruck & 10&10&10&10&10&10&6&6\\
		\hline
	\end{tabular} 
		\caption{Schleifstufen}
		\label{tab:Schleifstufen}
\end{table}

Zwischen jeder Körnung werden die Proben $3 min$ im Ultraschallbad in einer Seifenlauge gereinigt, um größere Schneidkörner und Abrieb nicht zu verschleppen, und die Dauer des Schleifens um $30 s$ verlängert. 

Zum Polieren wird eine Wabenscheibe mit destilliertem Wasser und einer Poliersuspension bestehend aus Oxid-Polier-Suspension ($0,05 \mu m$) und Wasserstoffperoxid im Verhältnis 5:1 benetzt. Jede Minute wird Poliersuspension nachgegeben, um eine kontinuierliche Politur zu gewährleisten.

\begin{table}[h]
	\centering

	\begin{tabular}{|c|c|c|c|}
		\hline 
		Schritt & Druck [N] & Zeit [min] & Richtung \\ 
		\hline 
		1 & 7 & 5 & Gegenlauf \\ 
		\hline 
		2 & 5 & 2 & Gleichlauf \\ 
	\hline 
	\end{tabular} 
	\caption{Polierstufen}
	\label{tab:Polierstufen}
\end{table}

Die Proben werden nach jedem Schritt $4 min$ in einem Ethanolbad ultraschallgereinigt. Nach Beendigung beider Polierschritte wird die Wabenscheibe mit Spülmittel gesäubert und die Schritte eins und zwei wiederholt. Dieser Prozess wird solange wiederholt bis die Probenoberfläche frei von Riefen und Fremdpartikeln ist. Ist dies der Fall, wird im letzten Schritt die Probenoberfläche mit Spülmittel und anschließend mit Ethanol gereinigt und getrocknet. 


\subsection{Ätzen}

Im letzten Schritt der Probenpräparation werden die Oberflächen der Trennfläche geätzt. Die polierte Oberfläche der Proben reflektiert Licht nahezu gleichmäßig, wodurch das Gefüge der Legierung nicht zu erkennen ist. 
Die Proben werden in einem Ätzmedium nach Kroll $7s$ geätzt. Martensitische Proben werden länger geätzt, hier $10s$. 

\begin{table}[h]
	\centering
	\begin{tabular}{|c|c|}
		
		\hline 
		Destilliertes Wasser
		& $100ml$
		\\ 
		\hline 
		Salpetersäure (HNO_{3})	& $6ml$
		\\ 
		\hline 
		Flusssäure (HF) & $3ml$
		\\ 
		\hline 
	\end{tabular} 
	\caption{Ätzlösung nach Kroll}
	\label{tab:Ätz_Kroll}
\end{table}


\section{Untersuchung der Mikrostruktur}

\subsection{Lichtmikroskop:}

	Nach der Probenpräparation werden die Proben im Lichtmikroskop untersucht. Es wird das Zeiss AX10 mit folgenden Auflösungen 20x 50x 100x benutzt. Es werden Bilder aufgenommen mit unterschiedlichen Vergrößerung damit man die Phasen gut erkennen kann. Jede Wärmebehandlung bringt zu einer anderen Struktur. Diese wird untersucht, in dem die Bilder vergleicht werden. 
	
	Es wurde bei den jeweiligen Proben unterschiedliche Stelle fotografiert. Mit einer bestimmten Auflösung erkennt man die unterschiedlichen Phasen: $\alpha$-Phase, $\beta$ -Phase und Transformierte- $\beta$ -Phase. Die Phasen sind wegen der Filter bei unterschiede Helligkeit erkennbar. $\alpha$ sind hell und $\alpha$ + $\beta$ sind dunkel. Um den Kontrast zu verbessern, werden C-DIC Filtern benutzt, falls man Sie nicht erkennt.


\subsection{Rasterelektronenmikroskopie (REM)}

	Das Rastereletronenmikroskop wird auch benutzt um Bilder zu erzeugen. (PROGRAMME) Das Mikroskop schafft höhere Auflösungen in eine kurze Zeit zu übertragen. Mit Hilfe von magnetische Linsen, werden ausgestrahlten Elektronen aufgenommen.  Diese können Bilder erzeugen. Unteranderen kann das REM auch Informationen übers die Probe zu verarbeiten. Das Programm X schlägt vor welche Elementen in der Probe auftreten. Man kann also untersuchen aus welche Elemente der Phasendiagram der Probe besteht. Das Programm X schafft auch die Massenverhältnisse der Legierung zu zeigen, so genannte EDX Röntgenanalyse.   
	
	Mit dem Befehl Backscatter Electrons BSE erzeugt man eine Schwarz-Weiß-Bild. Weiterhin werden Bilder erzeugt mit unterschiedliche Farben, die das Massenverhältnis von Molybdän und Aluminium, zum Beispiel. Siehe Abbildung 14.
	
	Es werden in unterschiedliche Stelle Flächenanalyse gemacht. 


	\subsection{großes REM}
		
Für eine bessere Auflösung bei größere Vergrößerung des Bildes zu bekommen ist der Smart SEM LEO 1550 zur Verfügung. Eine Prozesskammer ist motorisiert auf 5 Freiheitsgrad (X-, Y-, Z-Richtung, Neigung, Rotation) mit einer Luftschleuse. Das Programm Gemini steuert den REM. Die Bilder sind entweder mit sekundäre Elektronen (SE2) Detektoren erzeugt, oder mit Inlens (hohe Auflösung) Detektor. 

Es werden Bilder aufgenommen bei verschiedene Stelle der Probe. Mittelbereich und am Rand werden untersucht, mit Vergrößerungen von 20 000 – Fach. 



	
\section{Quantitative Analyse der Mikrostruktur}


	\subsection{Alpha-Primär Volumenanteil Analyse}
	
	Die Bilder die vom Lichtmikroskop erstellt wurde, werden mit Hilfe ein Programm analysiert. Gemessen wir wie viel $\alpha$ - p Anteil an die Gefüge ist. Ein Programm generiert die Volumenanteile bei den jeweiligen Aufnahmen. Für jede Probe werden mehrere Bilder analysiert, damit eine mittlere Prozentual berechnet wird. 
	



\section{Mechanische Prüfverfahren}

\subsection{Härteprüfung}

Die Härte der Proben wurde mit einer Vickers-Prüfung ermittelt. Bei der Vickers-Prüfung wird die Eindringhärte des Materials gegenüber eines Eindringkörpers in Form einer gleichseitigen Diamantpyramide gemessen. Die Diamantpyramide hat einen Öffnungswinkel von $136^\circ$ zwischen den Seitenflächen und wird mit $10 kg$ ($98,1 N$) statischem Druck $15 s$ lang in die Probe gedrückt. Die Längen der Diagonalen $d_1$ und $d_2$ des dabei entstehenden Eindrucks werden mittels einer optischen Messeinheit vermessen. Daraus lässt sich aus

\begin{equation}
HV=\frac {2*0,102*F*\sin \left( \frac{136^\circ}{2}\right) } {d^2} \approx 0,1891 \frac{F}{d^2}
	\label{eq:HV}
\end{equation}

mit der Kraft $F$ in Newton und $d=\frac {d_1 + d_2}{2} $ die Vickershärte $HV$ berechnen. 

\subsection{Zugversuch}
Zur Bestimmung wichtiger Werkstoffkennwerte wie der Bruchdehnung, Zugfestigkeit, Dehngrenze und des Elastizitätsmoduls werden Zugversuche durchgeführt. Der Zugversuch ist ein genormtes Standardverfahren, das zu den quasistatischen, zerstörenden Prüfverfahren gehört. In Größe und Form genormte Proben werden dabei mit geringer Geschwindigkeit bis zum Bruch gedehnt. Gleichzeitig werden die Längenänderung $\Delta L$ und die Kraft $F$ an der Probe gemessen. Mit der Anfangslänge $L_0$ und dem Anfangsquerschnitt $S_0$ lassen sich Nennspannung $\sigma$ und die Dehnung $\epsilon$ berechnen.

\begin{equation}
	\sigma=\frac{F}{S_0} ~~~~
	\epsilon=\frac{\Delta L} {L_0}
\end{equation}

Die Nennspannung und Dehnung werden in einem Spannungs-Dehnungs-Diagramm gegeneinander aufgetragen. 
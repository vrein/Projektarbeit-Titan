\chapter{Experimentelle Methoden}

\section {Metallografische Präparation (VR)}

\subsection{Trennen}
Die wärmebehandelten Proben werden in der Mitte mit einer Siliziumkarbid-Scheibe unter ständigem Kühlmittelfluss getrennt (Trennmaschine Jean Wirtz CUTO 20). Durchgehende Kühlung  verhindert eine zusätzliche, ungewollte Gefügeveränderung an der Schnittfläche während des Trennvorgangs.


\subsection{Einbetten}
Die getrennten Proben werden in Warmeinbettpressen (Buehler Simplimet Mounting Press 4000) für bessere Handhabung und Stützung der Randzonen eingebettet. Beim Warmeinbetten wird mit Hilfe von Druck und Temperatur die Probe in ein Kunststoffgranulat eingeschlossen. Vorteile des Warmeinbetten sind die hohe Härte und Spaltfreiheit des Einbettmaterials. Dabei wird Epomet als erste Schicht im Bereich der Probenoberfläche benutzt und für die oberflächenfernen Bereiche Bakelit, da Epomet eine bessere Spaltfüllung hat. Das Warmeinbetten erfolgte bei 180$^\circ$C und 3bar. 
Die fertig eingebetteten Proben werden entgratet und auf der Seite der Probenoberfläche mit einer Fase versehen.  


\subsection{Schleifen/Polieren}

Die Trennfläche der Proben wird in Vorbereitung auf die Ätzung der Oberfläche geschliffen und poliert. Ziel ist eine Oberfläche, die frei von Riefen und Fremdpartikeln ist. Als Schleif-/Poliergerät wurde ein ATM Saphir 550 benutzt.
Im ersten Schritt werden die Proben mit steigender Körnung im Gegenlauf geschliffen und dabei wassergekühlt (siehe Tab. \ref{tab:Schleifstufen}). 

\begin{table}[h]
	\centering
	\begin{tabular}{|c|c|c|c|c|c|c|c|c|}
		
		\hline 
		Körnung (FEPA P) & 180 & 240 & 320 & 400 & 600 & 800 & 1200 & 2500 \\ 
		\hline 
		Zeit [min:s] & 0:30 & 1:00 & 1:30 & 2:00 & 2:30 & 3:00 & 3:30 & 4:00 \\ 
		\hline 
		Anpressdruck [N] & 10&10&10&10&10&10&6&6\\
		\hline
	\end{tabular} 
	\caption{Schleifstufen}
	\label{tab:Schleifstufen}
\end{table}

Zwischen jeder Körnung werden die Proben $3 min$ im Ultraschallbad in einer Seifenlauge gereinigt, um größere Schneidkörner und Abrieb nicht zu verschleppen, und die Dauer des Schleifens um $30 s$ verlängert. 

Zum Polieren wird eine Wabenscheibe mit destilliertem Wasser und einer Poliersuspension bestehend aus Oxid-Polier-Suspension ($0,05 \mu m$) und Wasserstoffperoxid im Verhältnis 5:1 benetzt. Jede Minute wird Poliersuspension nachgegeben, um eine kontinuierliche Politur zu gewährleisten.

\begin{table}[h]
	\centering
	
	\begin{tabular}{|c|c|c|c|}
		\hline 
		Schritt & Druck [N] & Zeit [min] & Richtung \\ 
		\hline 
		1 & 7 & 5 & Gegenlauf \\ 
		\hline 
		2 & 5 & 2 & Gleichlauf \\ 
		\hline 
	\end{tabular} 
	\caption{Polierstufen}
	\label{tab:Polierstufen}
\end{table}

Die Proben werden nach jedem Schritt $4 min$ in einem Ethanolbad ultraschallgereinigt. Nach beiden Polierschritten wird die Wabenscheibe mit Spülmittel gesäubert und die Schritte 1 und 2 wiederholt. Es wird solange poliert bis die Probenoberfläche frei von Riefen und Fremdpartikeln ist. Im letzten Schritt wird die Probenoberfläche mit Spülmittel und anschließend mit Ethanol gereinigt und getrocknet. 


\subsection{Ätzen}

Im letzten Schritt der Probenpräparation werden die Oberflächen der Trennfläche geätzt. Die polierte Oberfläche der Proben reflektiert Licht nahezu gleichmäßig, wodurch das Gefüge der Legierung nicht zu erkennen ist. 
Die Proben werden in einem Ätzmedium nach Kroll $7s$, martensitische Proben $10s$ geätzt. 

\begin{table}[h]
	\centering
	\begin{tabular}{|c|c|}
		
		\hline 
		Destilliertes Wasser
		& $100ml$
		\\ 
		\hline 
		Salpetersäure ($HNO_{3}$)	& $6ml$
		\\ 
		\hline 
		Flusssäure ($HF$) & $3ml$
		\\ 
		\hline 
	\end{tabular} 
	\caption{Ätzlösung nach Kroll}
	\label{tab:Ätz_Kroll}
\end{table}



\section{Untersuchung der Mikrostruktur (TJ)}

\subsection{Lichtmikroskop}

Nach der Probenpräparation werden die Proben unter dem Lichtmikroskop (Zeiss AX10) untersucht. Es werden Bilder aufgenommen mit 200-facher bis 1000-facher Vergrößerung. Anhand der Bilder kann die Mikrostruktur der Probe erfasst werden. Die einzelnen Phasenanteile können mit Hilfe der verschiedenen Graustufen differenziert und analytisch ausgewertet werden. Dazu gehört die Korngrößen- und Phasenanteilbestimmung. 
Zusätzlich können Filter eingesetzt werden, um bestimmte Mikrostrukturen besser hervorzuheben.

\subsection{Rasterelektronenmikroskopi (REM)}

Das Rasterelektronenmikroskop wird benutzt, um eine dreidimensionale Darstellung der Oberfläche zu erzeugen. Das Mikroskop ermöglicht höhere Auflösungen und bietet die Möglichkeit, Oberflächen, Material sowie chemische Eigenschaften zu analysieren. Mit Hilfe von magnetische Linsen, werden ausgestrahlten Elektronen aus der Probeoberfläche aufgenommen. Diese können Bilder erzeugen. Unter anderen kann das REM mehrere Informationen über die Probe verarbeiten. Das Programm schlägt vor, welche Elementen in der Probe auftreten. Man kann also untersuchen, aus welche Elementen das Phasendiagramm der Probe besteht. Das Programm ist auch in der Lage die Massenverhältnisse in der Legierung anzuzeigen, die so genannte EDX Röntgenanalyse.   

Mit dem Befehl Backscatter Electrons BSE erzeugt man ein Schwarz-Weiß-Bild. Weiterhin werden Bilder mit unterschiedliche Farben erzeugt, welche zum Beispiel das Massenverhältnis von Molybdän und Aluminium wiedergeben. Siehe Abbildung 14.

Es werden an unterschiedliche Stellen Flächenanalyse erstellt. 

\subsection{großes REM}

Um zu einer besseren Auflösung bei großer Vergrößerung des Bildes zu bekommen, steht der Smart SEM LEO 1550 zur Verfügung. Er besitzt eine motorisierte Prozesskammer mit 5 Freiheitsgraden (X-, Y-, Z-Richtung, Neigung, Rotation), und einer Luftschleuse. Das Programm Gemini steuert den REM. Die Bilder sind entweder mit sekundär Elektronen Detektoren (SE2) erzeugt, oder mit dem Inlens Detektor (hohe Auflösung). 

Es werden Bilder an verschiedenen Stellen der Probe aufgenommen. Im Mittelbereich und am Rand wird mit 20000-facher Vergrößerung untersucht. 


\subsection{$\alpha_{p}$-Volumenanteil Analyse}

Die Bilder die vom Lichtmikroskop erstellt wurden, werden mit Hilfe eines Programmes analysiert. Gemessen wird wie viel $\alpha_{p}$ Anteil im Gefüge enthalten ist. Ein Programm generiert die Volumenanteile bei den jeweiligen Aufnahmen. Für jede Probe werden mehrere Bilder analysiert, damit ein Mittelwert berechnet werden kann. 



\section{Mechanische Prüfverfahren (VR)}

\subsection{Härteprüfung}

Die Härte der Proben wurde mit einer Vickers-Prüfung ermittelt. Bei der Vickers-Prüfung wird die Eindringhärte des Materials gegenüber eines Eindringkörpers in Form einer gleichseitigen Diamantpyramide gemessen. Die Diamantpyramide hat einen Öffnungswinkel von $136^\circ$ zwischen den Seitenflächen und wird mit $10 kg$ ($98,1 N$) statischem Druck $15 s$ lang in die Probe gedrückt. Die Längen der Diagonalen $d_1$ und $d_2$ des dabei entstehenden Eindrucks werden mittels einer optischen Messeinheit vermessen. Die automatische Vermessung durch die Software wird durch den Bediener bestätigt oder angepasst. Dadurch kann eine Genauigkeit bis auf 3\% erzielt werden. Daraus lässt sich aus

$$HV=\frac {2*0,102*F*\sin \left( \frac{136^\circ}{2}\right) } {d^2} \approx 0,1891 \frac{F}{d^2}$$

mit der Eindruckkraft $F$ in Newton und $d=\frac {d_1 + d_2}{2} $ die Vickershärte $HV$ berechnen. 

\subsection{Zugversuch}
Zur Bestimmung wichtiger Werkstoffkennwerte wie der Bruchdehnung, Zugfestigkeit, Dehngrenze und des Elastizitätsmoduls werden Zugversuche durchgeführt. Der Zugversuch ist ein genormtes Standardverfahren, das zu den quasistatischen, zerstörenden Prüfverfahren gehört. In Größe und Form genormte Proben werden dabei mit geringer Geschwindigkeit bis zum Bruch gedehnt. Gleichzeitig wird die Längenänderung $\Delta l$ und die Kraft $F$ an der Probe gemessen. Mit der Anfangslänge $l_0$ und dem Anfangsquerschnitt $S_0$ lassen sich Nennspannung $\sigma$ und die Dehnung $\epsilon$ berechnen.

$$\sigma=\frac{F}{S_0}$$

$$\epsilon=\frac{\Delta l} {l_0}$$

Die Nennspannung und Dehnung werden in einem Spannungs-Dehnungs-Diagramm gegeneinander aufgetragen. 
\chapter{Experimentelle Methoden}

\section {Metallografische Präparation}


\subsection{Ofenbehandlung }
Alle Proben werden unter Normalatmosphäre wärmebehandelt. Da für die mechanischen Eigenschaften 



\subsection{Trennen}
Die wärmebehandelten Proben werden in der Mitte mit einer Siliziumkarbid-Scheibe unter ständigem Kühlmittelfluss getrennt (Trennmaschine Jean Wirtz CUTO 20). Durchgehende Kühlung  verhindert eine zusätzliche, ungewollte Gefügeveränderung an der Schnittfläche während des Trennvorgangs.


\subsection{Einbetten}
Die getrennten Proben werden in Warmeinbettpressen (Buehler Simplimet Mounting Press 1000/4000) für besseres Handling und Stützung der Randzonen eingebettet. Beim Warmeinbetten wird mit Hilfe von Druck und Temperatur die Probe in ein Kunststoffgranulat eingebettet. Vorteile des Warmeinbettens sind die hohe Härte und Spaltfreiheit des Einbettmaterials. Dabei wird Epomet als erste Schicht im Bereich der Probenoberfläche benutzt und für die oberflächenfernen Bereiche Bakelit, da Epomet eine bessere Spaltfüllung hat. Das Warmeinbetten erfolgte bei $180 \deg C$ und $3 bar$. 
Die fertig eingebetteten Proben werden entgratet und auf der Seite der Probe mit einer Fase versehen.  


\subsection{Schleifen/Polieren}

Die Trennfläche der Proben wird in Vorbereitung auf die Ätzung der Oberfläche geschliffen und poliert. Ziel ist eine Oberfläche, die frei von Riefen und Fremdpartikeln ist. Als Schleif-/Poliergerät wurde ein ATM Saphir 550 benutzt.
Im ersten Schritt werden die Proben mit steigender Körnung im Gegenlauf geschliffen und dabei wassergekühlt (\ref{tab:Schleifstufen}). 

\begin{table}[h]
	\centering
	\begin{tabular}{|c|c|c|c|c|c|c|c|c|}
		
		\hline 
		Körnung (FEPA P) & 180 & 240 & 320 & 400 & 600 & 800 & 1200 & 2500 \\ 
		\hline 
		Zeit (min:s) & 0:30 & 1:00 & 1:30 & 2:00 & 2:30 & 3:00 & 3:30 & 4:00 \\ 
		\hline 
		Anpressdruck & 10&10&10&10&10&10&6&6\\
		\hline
	\end{tabular} 
		\caption{Schleifstufen}
		\label{tab:Schleifstufen}
\end{table}

Zwischen jeder Körnung werden die Proben $3 min$ im Ultraschallbad in einer Seifenlauge gereinigt, um größere Schneidkörner und Abrieb nicht zu verschleppen, und die Dauer des Schleifens um $30 s$ verlängert. 

Zum Polieren wird eine Wabenscheibe mit destilliertem Wasser und einer Poliersuspension bestehend aus Oxid-Polier-Suspension ($0,05 \mu m$) und Wasserstoffperoxid im Verhältnis 5:1 benetzt. Jede Minute wird Poliersuspension nachgegeben, um eine kontinuierliche Politur zu gewährleisten.

\begin{table}[h]
	\centering

	\begin{tabular}{|c|c|c|c|}
		\hline 
		Schritt & Druck [N] & Zeit [min] & Richtung \\ 
		\hline 
		1 & 7 & 5 & Gegenlauf \\ 
		\hline 
		2 & 5 & 2 & Gleichlauf \\ 
	\hline 
	\end{tabular} 
	\caption{Polierstufen}
	\label{tab:Polierstufen}
\end{table}

Die Proben werden nach jedem Schritt $4 min$ in einem Ethanolbad ultraschallgereinigt. Nach Beendigung beider Polierschritte wird die Wabenscheibe mit Spülmittel gesäubert und die Schritte eins und zwei wiederholt. Dieser Prozess wird solange wiederholt bis die Probenoberfläche frei von Riefen und Fremdpartikeln ist. Ist dies der Fall, wird im letzten Schritt die Probenoberfläche mit Spülmittel und anschließend mit Ethanol gereinigt und getrocknet. 


\subsection{Ätzen}

Im letzten Schritt der Probenpräparation werden die Oberflächen der Trennfläche geätzt. Die polierte Oberfläche der Proben reflektiert Licht nahezu gleichmäßig, wodurch das Gefüge der Legierung nicht zu erkennen ist. 
Die Proben werden in einem Ätzmedium nach Kroll $7s$ geätzt. Martensitische Proben werden länger geätzt, hier $10s$. 

\begin{table}[h]
	\centering
	\begin{tabular}{|c|c|}
		
		\hline 
		Destilliertes Wasser
		& $100ml$
		\\ 
		\hline 
		Salpetersäure (HNO_{3})	& $6ml$
		\\ 
		\hline 
		Flusssäure (HF) & $3ml$
		\\ 
		\hline 
	\end{tabular} 
	\caption{Ätzlösung nach Kroll}
	\label{tab:Ätz_Kroll}
\end{table}


\section{Untersuchung der Mikrostruktur}

\subsection{Lichtmikroskop:}
Nach der Probenpräparation werden die Proben im Lichtmikroskop untersucht. Es wird das Zeiss AX10 mit folgenden Auflösungen 1,25x 2,5x 5x 10x 20x 50x 100x benutzt. Es werden Bilder aufgenommen mit Vergrößerung ab 5-Fach bis 100-Fach. Die ersten beiden Vergrößerungen sind nicht ausgewählt, denn man erkennt die Korngrenzen nicht gut. Die Bilder werden benutzt um die Zusammensetzung der Proben zu vergleichen. 
Es wurde bei den jeweiligen Proben unterschiedliche Stelle fotografiert. Man erkennt die gleichsehende Korngroßen; sie haben ein gleichmäßiges Größenverhältnis. Mit einer bestimmten Auflösung erkennt man die unterschiedlichen Phasen, Alpha-Phase Beta-Phase und Transformierte-Beta-Phase. Die Phasen sind wegen der Filter bei unterschiede Helligkeit erkennbar. Helle Körne sind die Alpha-Phase und dunklere sind die Transformierte-Beta-Phase. Das liegt an die Lamellen das die T-B-Phase bei kleinere Auflösung dunkel sind. 


\subsection{Rasterelektronenmikroskopie (REM)}


Das Rastereletronenmikroskop ist ein Ersatz für den Lichtmikroskop. (Übers Programme schreiben; sind ja zwei Programme) Mit besseren Auflösung unter anderen Fähigkeiten... Es werden topografische, chemische und Materialuntersuchungen durchgeführt. Es werden mit Hilfe von magnetische Linsen Bilder erstellt. Elektronen beschleunigen sich zwischen einer Kathode und eine Anode. Wehneltzylinder dienen dazu den Strahl durch die Strahljustierspulen zu fließen, nur bei hohe Geschwindigkeiten. Je mehr Elektronen detektiert werden, desto heller wird das erzeugte Bild. Filtern wie C-DIC (heißt das so?) können auch benutzt werden, um eine bessere Kontrast im Bildschirm zu sehen.
Folgenden Aspekte treten auf:
Mit dem Backscatter Elektrons BSE Detektor werden Schwarz-Weiß-Bilder erzeugt. Diese sind wegen die reflektierte Elektronen erkennbar, da jedes Atom einen bestimmten Strahl emittiert. Man sieht am Bildschirm die Massenverhältnisse der Probe. BSE Modus wird oft benutz um die Phasen zu erkennen und unterscheiden.
Das erzeugte Bild ist nur möglich bei Strahlung von Sekundärelektronen SE. Sie werden von einem Detektor empfangen und als Bildinformation transformiert (umgewandelt?). 
Eine wichtige Aspekte der REM Mikroskope ist der energiedispersiven Röntgenanalyse EDX, auch EDS auf Englisch genannt. Es kann untersucht werden welche Elemente vortreten. Es finden sich Röntgenquanten statt. Elektronenbewegung treten an der Schale eines Atoms. Diese führt zu eine Energiestoß, die man messen kann. Jedes Element hat eine bestimmte Bewegungsart und kann damit identifiziert werden. Es wird mit einem Siliziumkristall gemessen. Das Programm schlägt dann vor welches Element des Periodensystems sein kann, bei Erkennung der Wellenlänge. 
Linien-, Punkt- und Flächenanalyse der Probeoberfläche sind im Programm zur Verfügung, wobei es wird nur die Flächenanalyse gebraucht. 
Größere Vergrößerung erbringen gute Auflösungen, im Gegensatz zu kleinere. Die große Menge an Informationen die gezeigt werden muss bei kleinere Auflösungen, sind schwierig zu übertragen. Es gibt eine Proportionsverhalten zwischen den Bildschirm und das gescannte Bild. Diese ist unabhängig vom Auflösung. Wichtig ist wie viel Information auf dem Bild gezeigt werden muss.  Bei einer kleinen Vergrößerung des Bildes sind weniger Information zu übertragen als bei einer größeren Vergrößerung. Es ist also möglich schnell den Ausschnitt einer Oberfläche näher heranholen.

%% Martensit Zerfall: 


%Weiterhin ist gewünscht die Härte der Legierung zu steigen. Dafür wurde eine Martensit Zerfall erwünscht. Dieses passiert so das eine martensitische Dekomposition im den Transus Phase herrscht.  
Das gesuchte Martensit Zerfall kommt nur im Transformierten-Beta-phase vor. Da wird den Martensit (a´) Phase in Alpha- und Beta-Phase umformen. Dadurch, dass Martensit Bildung sich im Nanometer Skala findet, erfolgen mehrere kleine Lamellen. Der Martensit ist also lokal im Gefüge zu finden, bei extrem kleine Diffusionsvorgänge. Es werden also nicht lange Zeiten gebrauch für die Vorgänge.
Die beste Versuchsprobe würde rausgewählt für den nächsten Vorgang. Die Probe 983°/1h/AC + 950/16min/WC ist ausgewählt worden, wegen ihre sichtbare Martensit Bildung in der Transformierten-Beta-Phase und die vorherige Härtesteigerung. Dazu wurde eine kleine Studie gemacht. Untersucht wird ob bei 580° und 610° ein Unterschied an der Härte ergibt. Wir haben die Temperaturen vom Short time Duplex aneal für Ti-64 als Vergleichsuntersuchung genommen. Erstmal ist dasselbe Temperatur genommen die am Paper vom T. Morita, K. Hatsuoka, T. Iizuka und K. Kawasaki gezeigt ist. Für die ersten Proben: 580° C. Und für die zweite Temperatur, sind 30K gestiegen (610° C). Untersucht wird, ob ein Unterschied bei einer höheren Temperatur gibt. Eine Erhöhung in der Temperatur hat dem Grund, dass die Legierung Ti-6242 eine Mollmenge von Molybdän von 2 Atome hat. In Vergleich zu Ti-64, dass eine Mollmenge von 4 V hat. Die Elemente sind vergleichbare beta-Stabilisatoren; nun hat Vanadium einen viel größeren Diffusionskoeffizienten als Molybdän. Um dieses auszugleichen, wird eine höhere Temperatur ausgewählt. Es besteht die Hoffnung eine Härtesteigerung zu erreichen. Für beide Schritte sind kurze Zeiten ausgewählt worden. Für die jeweiligen Temperaturen werden die Proben im Ofen für 8 Minuten bzw. 16 Minuten geglüht. Sie werden danach im Wasser abgekühlt. 
Es ist bekannt, dass bis das innere Teil der Probe die gewünschte Temperatur erreicht, brauch es eine gewisse Zeit. Diese Zeit wird mit 4 Minutengeschätzt. 
Für das Alpha ´ + Primär-Alpha wird auch ein Martensit Zerfall durchgeführt. Bei diesen Gefügen sieht das Vorgehen ein wenig anders aus. Dadurch das es sich global Martensit gebildet hat, werden hier höhere Zeiten ausgewählt. Für eine Temperatur von 610° werden 16 Minuten und 30 Minuten geschätzt. Es ist erwartet, dass die Proben mehr Zeit für den Zerfall brauchen, da es mehr Martensit gibt. Anders als bei Short Time Aneal. 
Die Temperaturen und jeweilige Zeiten wurden ausgewählt mit dem Ausgangspunkt der Three-Step-Short-Time-Aneal der Ti-64. Bei Ti-64 ist eine Temperatur von 580° gewählt. Wir haben 30° dazu gepackt denn unsere Legierung hat eine andere Beta-stabilisator, nämlich Molybdän. Dieses Element, anders als das Vanadium, hat eine größere Atomgröße. Diffusionsrate vom V ist größer als Mo Metastabil eingefroren. %





\section{Mechanische Prüfverfahren}

\subsection{Härteprüfung}

Die Härte der Proben wurde mit einer Vickers-Prüfung ermittelt. Bei der Vickers-Prüfung wird die Eindringhärte des Materials gegenüber eines Eindringkörpers in Form einer gleichseitigen Diamantpyramide gemessen. Die Diamantpyramide hat einen Öffnungswinkel von $136^\circ$ zwischen den Seitenflächen und wird mit $10 kg$ ($98,1 N$) statischem Druck $15 s$ lang in die Probe gedrückt. Die Längen der Diagonalen $d_1$ und $d_2$ des dabei entstehenden Eindrucks werden mittels einer optischen Messeinheit vermessen. Daraus lässt sich aus

\begin{equation}
HV=\frac {2*0,102*F*\sin \left( \frac{136^\circ}{2}\right) } {d^2} \approx 0,1891 \frac{F}{d^2}
	\label{eq:HV}
\end{equation}

mit der Kraft $F$ in Newton und $d=\frac {d_1 + d_2}{2} $ die Vickershärte $HV$ berechnen. 

\subsection{Zugversuch}
Zur Bestimmung wichtiger Werkstoffkennwerte wie der Bruchdehnung, Zugfestigkeit, Dehngrenze und des Elastizitätsmoduls werden Zugversuche durchgeführt. Der Zugversuch ist ein genormtes Standardverfahren, das zu den quasistatischen, zerstörenden Prüfverfahren gehört. In Größe und Form genormte Proben werden dabei mit geringer Geschwindigkeit bis zum Bruch gedehnt. Gleichzeitig werden die Längenänderung $\Delta L$ und die Kraft $F$ an der Probe gemessen. Mit der Anfangslänge $L_0$ und dem Anfangsquerschnitt $S_0$ lassen sich Nennspannung $\sigma$ und die Dehnung $\epsilon$ berechnen.

\begin{equation}

	\sigma=\frac{F}{S_0} ~~~~
	\epsilon=\frac{\Delta L} {L_0}
\end{equation}

Die Nennspannung und Dehnung werden in einem Spannungs-Dehnungs-Diagramm gegeneinander aufgetragen. 
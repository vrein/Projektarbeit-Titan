\chapter{Diskussion der Ergebnisse}

\section{Martensitbildung}


Im Gegensatz zu der zweiten Probenreihe, war  bei den ersten Proben, die bei 930°C für 8   oder 16 min WQ ..., wenig bis gar kein Martensit zu erkennen. Die Härte ist auch nur leicht gestiegen. Diese kleine Härteverbesserung zeigt aber trotzdem, dass  die Gefügestruktur  beeinflusst wurde. D.h es konnte sich bei 930°C das $\beta$ nicht schnell genug wachsen wie bei Ti64 oder neue $\beta$-Gebiete sind bereits in der kurzen Anlass-zeit  durch Diffusion von Mo stabilisiert worden. 
Das hat dazu geführt, dass sich, wenn überhaupt, nur in bestimmten Nano-Gebieten martensitische Strukturen gebildet haben. Außerdem ist diese Umwandlung nur bei  990/8/WQ-Probe und 960/8/WQ-Probe relativ besser zu sehen. Da die Randbereiche bei der Erwärmung zuerst durchwärmt werden, sind martensitische Strukturen auch am Rand besser erkennbar.
990-Proben haben durch die höhere Rekristallisationstemperatur den höchsten $\beta$-Anteil. Das erklärt, warum sich im Gegensatz zu 983- und 975-Proben zu Martensitbildung gekommen ist.
Bei den 960-Proben konnte sich aber trotzdem, auch wenn nur  lokal, Martensit bilden. Das ist möglicherweise darauf zurückzuführen, dass die Rekristallisationstemperatur  so niedrig war, dass sich Mo nicht vernünftig  in der $\beta$-Phase diffundieren konnte. Dadurch wurden nur beschränkte $\beta$-Gebiete stabilisiert.

Auch bei der Erwärmung für 16 min waren keine signifikante Änderung abzulesen. Das erklärt dass die Dauer des Anlassens bei 930°C nur einen geringen bis keinen Einfluss auf die Martensitbildung hat. 
Bei den anderen Proben hingegen ist die Härte von 344 HV auf 376 HV gestiegen. 
Das zeigt, dass 930°C zu niedrig  für die Diffusionskinetik von $\beta$ war. Das liegt wahrscheinlich daran, dass die Gleichgewichtstemperatur von Ti6242 höher ist als die von Ti64. Die Kinetische Energie von 930 reicht nicht aus um martensit zu bilden bei ti6242.

\paragraph{$\alpha$ + $\alpha^\prime$}

Bei 983°C  liegt die Legierung in dem  Zwei-Phasengebiet mit ca. 84\% $\beta$. Bei der Wasserabschreckung über $M_s$ wandelt sich die ganze $\beta$-Phase martensitisch um. Das hat dann zu einem Signifikanten Härteanstieg geführt. Duktilität ?


\section{Martensit-Zerfall}

Man sieht eine offensichtliche eine Erhöhung der Festigkeit. Das beste Versuchsprobe aus der 2. Wärmebehandlung ist bei 950° C / 16 min / WQ mit 376 HV gemessen worden. Man beobachtet eine Erhöhung von 17 bis 23 HV vom zweiten auf dritten Schritt. Damit kann man zufriedenweise sagen, dass eine gute Steigerung der Festigkeit bei Legierungsmaximierung stattgefunden hat. Nur daraus kann man mit große Wahrscheinlichkeit feststellen, dass eine martensitische Zerfall passiert ist. Es wird behauptet, dadurch dass $\beta$ weicher als $\alpha$ ist und dass mehr $\alpha$ im Gefüge zu finden sind, sollte das Vorgang eine Festigkeitszunahme bringen. 

Für den Parallelversuch ist keine Erhöhung zu sehen. Beim Glühen der Probe bei 983° C / 1h / Wasser gekühlt ist eine Härte von 405 HV. Leider ist bei den diesen Schritt die Festigkeit nicht gestiegen. Grund dafür könnte ja sein, dass es nicht genug Zeit gelassen worden ist. Es handelt sich um eine globale Struktur, es findet sich also viel mehr Martensit als im Three Step Short Time Anneal Legierung. 

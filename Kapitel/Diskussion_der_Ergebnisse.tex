\chapter{Diskussion der Ergebnisse}

\section{Martensitbildung (ZB)}


Im Gegensatz zu der dritten Probenreihe, war bei den zweiten Proben, die bei $930^\circ C$ für 8 bzw. 16 min geglüht und wasserabgeschreckt wurden, wenig bis gar kein Martensit zu erkennen. Die Härte ist auch nur leicht gestiegen. Diese kleine Härtesteigerung zeigt trotzdem, dass die Gefügestruktur beeinflusst wurde. D.h. bei $930^\circ C$ konnte das $\beta$ nicht schnell genug wachsen wie bei Ti-64 oder die neuen $\beta$-Gebiete sind bereits in der kurzen Anlasszeit durch Diffusion von Mo stabilisiert worden. 
Dies hat dazu geführt, dass sich, wenn überhaupt, nur in bestimmten Nanogebieten martensitische Strukturen gebildet haben. Nur bei der 990/8/WQ- und 960/8/WQ-Probe war eine Umwandlung eindeutig festzustellen. Da die Randbereiche bei der Wärmebehandlung zuerst erwärmt wurden, sind martensitische Strukturen am Rand besser erkennbar.
Die 990$^\circ$C-Probe hat durch die höhere Rekristallisationstemperatur den höchsten $\beta$-Anteil. Das erklärt, warum es im Gegensatz zu den 983$^\circ$C- und 975$^\circ$C-Proben zu Martensitbildung gekommen ist.
Bei der 960$^\circ$C-Probe konnte sich aber trotzdem, auch wenn nur lokal, Martensit bilden. Das ist möglicherweise darauf zurückzuführen, dass die Rekristallisationstemperatur so niedrig war, dass Mo nicht ausreichend in die $\beta$-Phase diffundieren konnte. Dadurch wurden die $\beta$-Gebiete nur beschränkt stabilisiert.

Auch bei der Erwärmung für 16 min war keine signifikante Veränderung im Gefüge feststellbar. Das erklärt, dass die Dauer des Anlassens bei $930^\circ C$ nur einen geringen bis keinen Einfluss auf die Diffusion hat. 
Bei den Proben hingegen, die bei 950$^\circ$C geglüht wurden, ist die Härte von 344 HV auf 376 HV gestiegen. 
Das zeigt, dass $930^\circ C$ zu niedrig für die Diffusionskinetik von $\beta$ war. Das liegt wahrscheinlich daran, dass die Gleichgewichtstemperatur von Ti-6242 höher ist als die von Ti-64. D.h. die Diffusionsenergie bei $930^\circ C$ reicht nicht aus, um die $\beta$-Lamellen innerhalb von 8 oder 16 $min$ signifikant wachsen zu lassen und Martensit zu bilden.

\paragraph{$\alpha_p$ - $\alpha^\prime$}

Bei 983$^\circ$C liegt die Legierung in dem  Zwei-Phasengebiet mit ca. 84 \% $\beta$ vor. Bei der Wasserabschreckung über $M_s$ wandelt sich die ganze $\beta$-Phase martensitisch um. Das führte zu einem signifikanten Härteanstieg. Der Ätzvorgang im Rahmen der Probenpräparation kann ebenfalls einen Einfluss die optische Analyse der Probenoberfläche haben. 
%Werden Martensitstrukturen nicht ausreichend lang geätzt

\section{Martensit-Zerfall (TJ)}

Man sieht eine offensichtliche Erhöhung der Festigkeit vom Zweiten auf den dritten Schritt. Die beste Versuchsprobe aus der 2. Wärmebehandlung ist bei $950\circ C$ /16 min/WQ mit 376 HV gemessen worden. Man beobachtet eine Erhöhung von 17 bis 23 HV. Damit kann man sagen, dass eine signifikante Steigerung der Festigkeit in der Legierung stattgefunden hat. Mit hoher Wahrscheinlichkeit ist dies auf den martensitischen Zerfall zurückzuführen. 

Da $\beta$ weicher als $\alpha$ ist, und im Gefüge mehr $\alpha$ zu finden war, wird vermutet, dass dies zur Festigkeitszunahme geführt hat.

Im Parallelversuch ist keine Erhöhung nachzuweisen gewesen. Beim Glühen der Probe bei $983\circ C$ /1h/WQ ist eine Härte von 405 HV entstanden. Leider ist bei diesen Schritt die Festigkeit nicht gestiegen. Grund dafür könnte sein, dass die Zeit für das Anlassen zu gering war. Bei dieser Probe handelt es sich um eine globale Struktur. In ihr findet sich viel mehr Martensit als in der vorherigen Probe.



\chapter{Diskussion der Ergebnisse}

\section{Martensitbildung}


Im Gegensatz zu der zweiten Probenreihe, war  bei den ersten Proben, die bei $930^\circ C$ für 8   oder 16 min WQ ..., wenig bis gar kein Martensit zu erkennen. Die Härte ist auch nur leicht gestiegen. Diese kleine Härteverbesserung zeigt aber trotzdem, dass  die Gefügestruktur  beeinflusst wurde. D.h es konnte sich bei $930^\circ C$ das $\beta$ nicht schnell genug wachsen wie bei Ti64 oder neue $\beta$-Gebiete sind bereits in der kurzen Anlass-zeit  durch Diffusion von Mo stabilisiert worden. 
Das hat dazu geführt, dass sich, wenn überhaupt, nur in bestimmten Nano-Gebieten martensitische Strukturen gebildet haben. Außerdem ist diese Umwandlung nur bei  990/8/WQ-Probe und 960/8/WQ-Probe relativ besser zu sehen. Da die Randbereiche bei der Erwärmung zuerst durchwärmt werden, sind martensitische Strukturen auch am Rand besser erkennbar.
990-Proben haben durch die höhere Rekristallisationstemperatur den höchsten $\beta$-Anteil. Das erklärt, warum sich im Gegensatz zu 983- und 975-Proben zu Martensitbildung gekommen ist.
Bei den 960-Proben konnte sich aber trotzdem, auch wenn nur  lokal, Martensit bilden. Das ist möglicherweise darauf zurückzuführen, dass die Rekristallisationstemperatur  so niedrig war, dass sich Mo nicht vernünftig  in der $\beta$-Phase diffundieren konnte. Dadurch wurden nur beschränkte $\beta$-Gebiete stabilisiert.

Auch bei der Erwärmung für 16 min waren keine signifikante Schwankungen abzulesen. Das erklärt dass die Dauer des Anlassens bei $930^\circ C$ nur einen geringen bis keinen Einfluss auf die Diffusion  hat. 
Bei den anderen Proben hingegen ist die Härte von 344 HV auf 376 HV gestiegen. 
Das zeigt, dass $930^\circ C$ zu niedrig  für die Diffusionskinetik von $\beta$ war. Das liegt wahrscheinlich daran, dass die Gleichgewichtstemperatur von Ti6242 höher ist als die von Ti64. D.h die  Diffusionsenergie bei $930^\circ C$ reicht nicht aus, um die  $\beta$-Lamellen innerhalb von 8 oder 16 $min$ schnell genug wachsen zu lassen. um martensit zu bilden bei ti6242.

\paragraph{$\alpha$ + $\alpha^\prime$}

Bei 983°C  liegt die Legierung in dem  Zwei-Phasengebiet mit ca. 84\% $\beta$. Bei der Wasserabschreckung über $M_s$ wandelt sich die ganze $\beta$-Phase martensitisch um. Das hat dann zu einem Signifikanten Härteanstieg geführt. 


\section{Martensit-Zerfall}

\section{Martensit-Zerfall}

Man sieht eine offensichtliche Erhöhung der Festigkeit vom Zweiten auf den dritten Schritt. Die beste Versuchsprobe aus der 2. Wärmebehandlung ist bei $950\circ C$ /16 min/WQ mit 376 HV gemessen worden. Man beobachtet eine Erhöhung von 17 bis 23 HV. Damit kann man sagen, dass eine signifikante Steigerung der Festigkeit in der Legierung stattgefunden hat. Mit hoher Wahrscheinlichkeit ist dies auf den martensitischen Zerfall zurückzuführen. 

Da $\beta$ weicher als $\alpha$ ist, und im Gefüge mehr $\alpha$ zu finden war, wird vermutet, dass dies zur Festigkeitszunahme geführt hat.

Im Parallelversuch ist keine Erhöhung nachzuweisen gewesen. Beim Glühen der Probe bei $983\circ C$ /1h/WQ ist eine Härte von 405 HV entstanden. Leider ist bei diesen Schritt die Festigkeit nicht gestiegen. Grund dafür könnte sein, dass die Zeit für das Anlassen zu gering war. Bei dieser Probe handelt es sich um eine globale Struktur. In ihr findet sich viel mehr Martensit als in der vorherigen Probe.



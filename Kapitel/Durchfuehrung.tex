\chapter{Durchführung}

\section{$\alpha_p$-Studie (VR)}
Zur Maximierung der Zugfestigkeit der Legierung Ti6242 wurde zuerst der Einfluss des $\alpha_p$-Phasenanteils auf die Zugfestigkeit untersucht. Laut \ref{bib:} konnte bei der Legierung IMI834 eine maximale Zugfestigkeit bei einem $\alpha_p$-Anteil von 10--20\% festgestellt werden. \\
Um eine größtmögliche Härtesteigerung gegenüber der as-received-Probe (AR) zu erzielen, wurden vier Proben bei unterschiedlichen Temperaturen $1h$ unter der $\beta_t$-Temperatur geglüht und anschließend luftgekühlt (AC: air cooled) (\ref{tab:alphap}). Dabei stellt sich ein bimodales Gefüge ein. Dieser Schritt wurde beim TS-STDA nicht explitit durchgeführt, da der erste Schritt dort gleichzeitig das bimodale Gefüge einstellt und die $\beta$-Phase martensitisch umwandelt. Die vier Proben wurden inklusive einer AR-Probe metallografisch präpariert und ausgewertet.



\begin{figure}[h]
	\includegraphics{Bilder/Phasendiagramm_beta.png}
	\caption{Phasendiagramm für Titan mit $\beta$-stabilisierenden Elementen}
	\label{fig:phadia}
\end{figure}



\begin{table}
		\centering
	\begin{tabular}{|c|c|c|c|}
	\hline 
	Probenbezeichnung & Temperatur [$\circ C$] & Zeit [$h$] & Abkühlmethode \\ 
	\hline 
	BM990 & 990 & 1 & AC\\ 
	\hline 
	BM983 & 983 & 1 & AC\\ 
	\hline 
	BM975 & 975 & 1 & AC\\ 
	\hline 
	BM960 & 960 & 1 & AC\\ 
	\hline 
	\end{tabular} 
	\caption{Wärmebehandlung der $\alpha_p$-Studie}
	\label{tab:alphap}
\end{table}


\section{Martensit-Bildung}

Um Martensit zu bilden werden Ti64-Teile  nach der ersten Wärmebehandlung, wie es in Abbildung \ref{STDA} zusammengefasst wird, für 1 min  bei 930°C erwärmt und dann  auf Raumtemperatur wassergekühlt. Unter dem Einfluss von der Diffusion soll sich die erhaltene und metastabile Beta Phase aus der bimodalen Struktur weiter wachsen. Die kurze Erwärmungszeit soll dafür sorgen, dass die neu gebildeten Beta-Gebiete nicht mit $\beta$-Stabilisatoren, in diesem Fall Vanadium, bereichert  und dadurch stabilisiert werden. Durch das schnelle Abschrecken auf Raumtemperatur wandelt sich das "neue "$\beta$ diffusionslos und lokal in Martensit um.

\begin{figure}[H]
	\centering
	\includegraphics[width=0.9\textwidth]{Bilder/ts-stda}
	\caption{Vorgehensweise nach dem Duplex-Anneal bei STDA für Ti-64 (Strengthening of Ti–6Al–4V Alloy by Short-Time Duplex Heat Treatment)}
	\label{STDA}
\end{figure}

Da die $\beta_{t}$ von Ti-64 relativ niedriger ist als die von Ti-6242, liegt auch ihrer Gleichgewichtstemperatur unterhalb der von Ti-6242. Außerdem hat Vanadium im Vergleich zu Molybdän eine viel größere Diffusionsrate in Titan, was die schnellen Anlasszeiten noch weiter erklärt[Titan und Titan legierungen, Zwicker]. Aus diesen Gründen wurden in diesem Schritt die Ti-6242-Proben nach dem Duplex-Anneal für 8 und 16 min jeweils bei 930°C und 950°C wärmebehandelt.

Eine bekannte Wärmebehandlung von $\alpha$+$\beta$-Titanlegierungen ist die  \textit{Solution treatment and quenching}, wobei die Titanlegierung direkt von einer Temperatur $T_{1}$ unterhalb  $\beta_{t}$ nach 0,5-1 h abgeschreckt wird. Wie bei der oben beschriebenen Wärmebehandlung stellt sich bei $T_{1}$ ein zweiphasiges Gefüge mit $\alpha_p$ und $\beta$ ein. Die $\beta$-Phase wandelt sich  dann auch beim Abschrecken martensitisch um und wird $\alpha^\prime$ genannt.(Strengthening of Ti–6Al–4V Alloy by Short-Time Duplex Heat Treatment)
Zum Vergleich zu der studierten Wärmebehandlung werden AR-Proben bei 983°C für 1h erwärmt und wassergekühlt.

\section{Martensit-Zerfall (TJ)}



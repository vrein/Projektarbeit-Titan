\begin{document}

\section {Metallografische Präparation}

\subsection{Ofenbehandlung }

\subsection{Trennen}
Die wärmebehandelten Proben werden in der Mitte mit einer Siliziumkarbid-Scheibe unter ständigem Kühlmittelfluss getrennt (Trennmaschine Jean Wirtz CUTO 20). Durchgehende Kühlung  verhindert eine zusätzliche, ungewollte Gefügeveränderung an der Schnittfläche während des Trennvorgangs.

 
\subsection{Einbetten}
Die getrennten Proben werden in Warmeinbettpressen (Buehler Simplimet Mounting Press 1000/4000) für besseres Handling und Stützung der Randzonen eingebettet. Beim Warmeinbetten wird mit Hilfe von Druck und Temperatur die Probe in ein Kunststoffgranulat eingebettet. Vorteile des Warmeinbettens sind die hohe Härte und Spaltfreiheit des Einbettmaterials. Dabei wird Epomet als erste Schicht im Bereich der Probenoberfläche benutzt und für die oberflächenfernen Bereiche Bakelit, da Epomet eine bessere Spaltfüllung hat. Das Warmeinbetten erfolgte bei 180 °C und 3 bar. 
Die fertig eingebetteten Proben werden entgratet und auf der Seite der Probe mit einer Fase versehen.  


\subsection{Schleifen/Polieren}
 
Die Trennfläche der Proben wird in Vorbereitung auf die Ätzung der Oberfläche geschliffen und poliert. Ziel ist eine Oberfläche, die frei von Riefen und Fremdpartikeln ist. Als Schleif-/Poliergerät wurde ein ATM Saphir 550 benutzt.
Im ersten Schritt werden die Proben mit steigender Körnung im Gegenlauf geschliffen und dabei wassergekühlt. 


\begin{tabular}{|c|c|c|c|c|c|c|c|c|}
	\hline 
	Körnung (FEPA P) & 180 & 240 & 320 & 400 & 600 & 800 & 1200 & 2500 \\ 
	\hline 
	Zeit (min:s) & 0:30 & 1:00 & 1:30 & 2:00 & 2:30 & 3:00 & 3:30 & 4:00 \\ 
	\hline 
	Anpressdruck & 10&10&10&10&10&10&6&6\\
	\hline
\end{tabular} 


Zwischen jeder Körnung werden die Proben im Ultraschallbad in einer Seifenlauge gereinigt, um größere Schneidkörner und Abrieb nicht zu verschleppen, und die Dauer des Schleifens um 30 s verlängert. 

Zum Polieren wird eine Wabenscheibe mit destilliertem Wasser und einer Poliersuspension bestehend aus Oxid-Polier-Suspension (0,05 Micron) und Wasserstoffperoxid im Verhältnis 5:1 benetzt. Jede Minute wird Poliersuspension nachgegeben, um eine kontinuierliche Politur zu gewährleisten. 

\begin{tabular}{|c|c|c|c|}
	\hline 
	Schritt & Druck [N] & Zeit [min] & Richtung \\ 
	\hline 
	1 & 7 & 5 & Gegenlauf \\ 
	\hline 
	2 & 5 & 2 & Gleichlauf \\ 
	\hline
	\label{key}
\end{tabular} 


Die Proben werden nach jedem Schritt in einem Ethanolbad ultraschallgereinigt. Nach Beendigung beider Polierschritte wird die Wabenscheibe mit Spülmittel gesäubert und die Schritte eins und zwei wiederholt. Dieser Prozess wird solange wiederholt bis die Probenoberfläche frei von Riefen und Fremdpartikeln ist. Ist dies der Fall, wird im letzten Schritt die Probenoberfläche mit Spülmittel und anschließend mit Ethanol gereinigt und getrocknet. 


\subsection{Ätzen}

Im letzten Schritt der Probenpräparation werden die Oberflächen der Trennfläche geätzt. Die polierte Oberfläche der Proben reflektiert Licht nahezu gleichmäßig, wodurch das Gefüge der Legierung nicht zu erkennen ist. 
Die Proben werden in einem Ätzmedium nach Kroll sieben sekunden geätzt. Martensitische Proben werden länger angeätzt, hier 10s. 

\begin{tabular}{|c|c|}
	\hline 
	Destilliertes Wasser
	& $100ml$
	\\ 
	\hline 
	Salpetersäure ($HNO_{3}$)	& $6ml$
	\\ 
	\hline 
	Flusssäure (HF) & $3ml$
	\\ 
	\hline 
\end{tabular} 


\subsection{Härteprüfung}

Die Härte der Proben wurde mit einer Vickers-Prüfung ermittelt. Bei der Vickers-Prüfung wird die Eindringhärte des Materials gegenüber eines Eindringkörpers in Form einer gleichseitigen Diamantpyramide gemessen. Die Diamantpyramide wird mit statischem Druck 15 sekunden in die Probe gedrückt. Die Längen der Diagonalen des dabei entstehenden Eindrucks werden mittels einer optischen Messeinheit vermessen. Daraus lässt sich 

\end{document}
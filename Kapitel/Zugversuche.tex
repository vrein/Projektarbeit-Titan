\chapter{Zugversuche }

(ZB)

\section{Durchführung}

Um die Einflüsse der Wärmebehandlungen genauer betrachten zu können, wurden weitere wichtige mechanische Kennwerte wie die Duktilität, Zugfestigkeit und Bruchdehnung von 8 Probendurch einen Zugversuch ermittelt. \\
Um auch den Einfluss von dem Martensitzerfall in der Strategie 1 auf die Duktilität genauer diskutieren zu können, wird auch Probe 4, siehe Tabelle \ref{tab:ubersicht} betrachtet.


\begin{table}[h]
	\centering
	\begin{tabular}{|c|c|}
		\hline 
		Probe & Wärmebehandlung \\ 
		\hline 
		1 & AR1 \\ 
		\hline 
		2 & AR2 \\ 
		\hline 
		3 &  983$^\circ$C/1h/AC + 950$^\circ$C/16min/WQ + 610$^\circ$C/16min/AC \\ 
		\hline 
		4 &  983$^\circ$C/1h/AC + 950$^\circ$C/16min/WQ + 610$^\circ$C/16min/AC \\ 
		\hline 
		5 &  983$^\circ$C/1h/WQ + 610$^\circ$C/30min/AC \\ 
		\hline 
		6 &  983$^\circ$C/1h/WQ + 610$^\circ$C/30min/AC \\ 
		\hline 
		7 &  983$^\circ$C/1h/AC + 950$^\circ$C/16min/WQ \\ 
		\hline 
		8 &  983$^\circ$C/1h/AC + 950$^\circ$C/16min/WQ \\ 
		\hline 
	\end{tabular}
	\label{tab:ubersicht} 
\end{table}

\section{Ergebnisse}

\begin{table}
	\centering
	\begin{tabular}{|c|c|c|c|c|c|c|c|}
		\hline
		Probe & $d_0$ [mm] & $S_0$ [mm$^2$] & E [GPa] & $R_{p0,2}$ [MPa]& $R_m$ [MPa]& $A_g$ [\%]& $A$ [\%]\\
		\hline
		1 & 5,06 & 20,11 & 128 & 956 & 1019&4,8&17,9 \\
		\hline
		2 &5,04&19,95&122&946&1013&5,1&15,7\\
		\hline
		3 & 5,11&20,51& 122&1015&1085&2,1&3,0\\
		\hline
		4 &5,16& 20,91& 117 & 1007& 1090& 1,7&  1,9 \\
		\hline
		5&5,04 &19,95& 120& 952& 1033& 1,9 &2,5\\
		\hline
		6 &5,02& 19,79& 124& 968& 1023 &1,0 & 2,0\\
		\hline
		7&5,17& 20,99& 113& 925& 1042& 1,9& 2,1\\
		\hline
		8 & 5,12 & 20,59 & 118 & 952 & 1063 & 1,3 & 1,4\\
		\hline
	\end{tabular}
	\label{tab:zugversuche}
	\caption{Messwerte der Zugversuche bei 23,3$^\circ$ C Raumtemperatur}
\end{table}

\section{Diskussion der Ergebnisse}
AR-Proben haben ein globulares Gefüge mit $\alpha$ und $\beta$-Phasen(Siehe Abbildung \ref{fig}). Diese grobe Gefügestruktur mit 48\% stabilem $\beta$ führt zu Erhöhung der Duktilität und Abnahme der Festigkeit. Das erklärt die höheren Dehnungen von Probe 1 und 2 im Vergleich zu den anderen Proben.

\paragraph{Strategie 1}
Nach dem 2. Schritt (Kapitel \ref{MB1}) hat sich Martensit in der Transformierten $\beta$-Phase des bimodalen Gefüges gebildet. Die feinere Gefügestruktur  durch die dünnen Martensitnadeln und die feinen $\alpha$-Lamellen  haben nur zu einer leichten Festigkeitssteigerung bei Probe5 und 6 geführt. Des weiteren hat die Duktilität stark  abgenommen.
Nach dem 3. Schritt (Kapitel \ref{MZ1}) ist die Zugfestigkeit der Proben 3 und 4 durch die weitere Verfeinerung des Gefüges weiter gestiegen. Im Vergleich zu Proben 5 und 6 , hat die Duktilität bei Proben 3 leicht zugenommen. Das ist  darauf zurückzuführen, dass sich  $\alpha$ und  $\beta$ Phasen durch den partiellen Martensitzerfall gebildet haben.


\paragraph{Strategie 2}
Proben 7 und 8 zeigen gegenüber die anderen Proben eine relativ hohe Zugfestigkeit.  Dies ist durch die feinen Martensitplatten, die ca. 80\% der Probenoberfläche repräsentieren, zu erklären.
Mit einer längeren Anlasszeit bzw. einer höheren Temperatur im 
2. Schritt ist eine höere Duktilität durch die Einstellung eines gröberen Gefüges  zu erwarten.
\newline

















